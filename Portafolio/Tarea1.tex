\documentclass{article}
\usepackage{graphicx} % Required for inserting images

\title{Tarea 1 - Fundamentos del Desarrollo Móvil (Arquitectura movil)}
\author{Jorge Rafael Robledo Ramírez}
\date{Febrero 2024}

\begin{document}
\maketitle

Los fundamentos del desarrollo móvil, más que nada la arquitectura, es como se conforma una aplicacion movil para su funcionamiento, algunos de estos puntos son:

\begin{itemize}
    \item \textbf{Arquitectura cliente-servidor:} las aplicaciones siguen una arquitectura cliente-servidor, donde el usuario se comunica con un servidor para obtener la datos o enviar datos que requiere el usuario y se ejecute en el dispositivo.

    \item \textbf{Sistema operativo movil:} los dispositivos ustilizan sistemas operativos, como por ejemplo Android o IOS, cada uno tiene sus caracteristicas y funciones.

    \item \textbf{Desarrollo nativo vs desarrollo multiplataforma:} en las apps existen las aplicaciones nativas las cuales estan solamente para cierto sistema operativo y son desarrolladas mediante cierto lenguaje de programacion a diferencia de las multiplataformas que son utilizados con un lenguaje de programacion distinto y las apps creadas pueden ser utilizadas en varios sistemas operativos.

    \item \textbf{Interfaz de usuario (UI) y experiencia de usuario (UX):} la interfaz de usuario se refiere mas que nada al diseño visual que el usuario ve y la experiencia es la interaccion con la aplicacion.

    \item \textbf{API's (Interfaces de programacion de aplicaciones):} las API´s permiten al desarrollador utilizar servicios de otras empresas con el fin de un software mas confiable y facil, las API´s pueden ser tanto funcionalidades como mapas, comentarios, etc. al igual datos de base de datos pero sin interacturar directamente con las base de datos y cierta informacion de las empresas.

    \item \textbf{Persistencia de datos:} Las aplicaciones utilizan costantemente datos que almacenar, para ello hay diferentes formas, ya sea mediante base de datos incorporados, archivos locales o servicios en la nube.

    \item \textbf{Ciclo de vida de una aplicacion:} comprender el ciclo de vida de una aplicacion movil es crucial. esto abarca desde la incializacion hasta la suspension y cierre de la aplicacion. aprovechar eventos especificos del ciclo de vida permite gestionar recursos de manera eficiente. 

    \item \textbf{Pruebas y depuracion:} la calidad de la aplicacion es muy importante, siempre se deben de hacer pruebas para asi evitar fallos y errores en el programa y asi darle al usuario una mala experiencia.
    
    \item \textbf{Actualizaciones y mantenimiento:} las aplicaciones deben de estar en constantes actualizaciones y mantenimeintos, ya sea para agregar contenido o mejorar aspectos tanto de diseño como de seguridad.
\end{itemize}
\end{document}
