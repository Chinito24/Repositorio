\documentclass{article}
\usepackage{graphicx} % Required for inserting images

\title{Tarea 2 - Aplicaciones nativas y no nativas}
\author{Jorge Rafael Robledo Ramirez}
\date{February 2024}

\begin{document}

\maketitle Las aplicaciones nativas y no nativas se refieren a cómo están desarrolladas y optimizadas para ejecutarse en un determinado entorno o plataforma. Aquí hay una breve explicación de ambos tipos:

\textbf{Aplicaciones nativas:} las aplicaciones nativas son las que se hacen con cierto lenguaje de programacion exclusivamente para cierto dispositivo, como por ejemplo IOS uno de los lenguajes que se utiliza es Swift u Object-C y para Android son principalmente Java, Kotlin, entre otros.
  \begin{itemize}
    \item \textbf{Ventajas:} las ventajas que tiene es que al ser nativas su rendimiento es mucho mejor a comparacion de hacerlas con otros lenguajes, y la experiencia es mucho mayor.

    \item \textbf{Desventajas:} las desventajas es que puede llegar a ser mas costoso el desarrollo de las apps y el mantenimiento puede ser mas complejo.

  \end{itemize}

\textbf{Aplicaciones no nativas:} las aplicaciones no nativas estan echar para que la aplicacion pueda utilizarse en cualquier sistema operativo.

  \begin{itemize}
    \item \textbf{Ventajas:} las ventajas es que no se ocuparia la creacion de cero para la implementacion a otro sistema operativo solo adaptarlo, ademas que los costos son menos que los nativos.

    \item \textbf{Desventajas:} las desventajas es que el rendimiento puede ser menor y al no ser nativa algunas funciones no pueden estar disponibles para lenguajes no nativos.

  \end{itemize}
\end{document}
