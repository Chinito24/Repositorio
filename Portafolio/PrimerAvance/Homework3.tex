\documentclass{article}
\usepackage{graphicx} % Required for inserting images

\title{Homework 1 - Fundamentals of Mobile Development (Mobile Architecture)}
\author{Jorge Rafael Robledo Ramírez}
\date{Febrero 2024}

\begin{document}

\maketitle % Agrega el título, autor y fecha según la información proporcionada

\begin{enumerate}

    \item \textbf{Model-View-Controller (MVC):}
        \begin{itemize}
            \item \textbf{Description:} Separation of business logic (Model), user interface (View), and control logic (Controller).
            \item \textbf{Application:} Useful for organizing code and facilitating maintainability.
        \end{itemize}

    \item \textbf{Model-View-ViewModel (MVVM):}
        \begin{itemize}
            \item \textbf{Description:} Similar to MVC but with a special focus on separating view logic and presentation logic through a ViewModel.
            \item \textbf{Application:} Especially effective when working with frameworks that support data binding, such as Android with Jetpack.
        \end{itemize}

    \item \textbf{Singleton:}
        \begin{itemize}
            \item \textbf{Description:} Ensures that a class has only one instance and provides a global point of access to that instance.
            \item \textbf{Application:} Useful for managing shared resources, such as a local database or a network service.
        \end{itemize}

    \item \textbf{Observer:}
        \begin{itemize}
            \item \textbf{Description:} Defines a one-to-many dependency between objects, so that when one object changes state, all its dependents are notified and updated automatically.
            \item \textbf{Application:} Useful for implementing real-time updates, such as notifications or data updates.
        \end{itemize}

    \item \textbf{Facade:}
        \begin{itemize}
            \item \textbf{Description:} Provides a unified interface for a set of interfaces in a subsystem, simplifying complexity.
            \item \textbf{Application:} Can be used to simplify interactions with complex services or modules.
        \end{itemize}

    \item \textbf{Adapter:}
        \begin{itemize}
            \item \textbf{Description:} Allows incompatible interfaces to work together.
            \item \textbf{Application:} Useful when integrating new features or libraries into an existing application.
        \end{itemize}

    \item \textbf{Prototype:}
        \begin{itemize}
            \item \textbf{Description:} Creates new objects by copying an existing object, known as the prototype.
            \item \textbf{Application:} Useful when there is a need for creating complex and costly objects.
        \end{itemize}

    \item \textbf{Builder:}
        \begin{itemize}
            \item \textbf{Description:} Separates the construction of a complex object from its representation, allowing the creation of different representations.
            \item \textbf{Application:} Useful when working with objects that have multiple possible configurations.
        \end{itemize}

\end{enumerate}

\end{document}

