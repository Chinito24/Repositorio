\documentclass{article}
\usepackage{graphicx} % Required for inserting images

\title{Tarea 2 - native applications and non-native applications}
\author{Jorge Rafael Robledo Ramirez}
\date{February 2024}

\begin{document}

\maketitle Native and non-native applications refer to how they are developed and optimized to run on a specific environment or platform. Here is a brief explanation of both types::

\textbf{natives applications:} Native applications are those developed using a specific programming language exclusively for a particular device. For example, for iOS, one of the languages used is Swift or Objective-C, while for Android, they are primarily Java, Kotlin, among others.
  \begin{itemize}
    \item \textbf{Advantages} The advantages of native applications lie in their superior performance compared to those developed using other languages. The user experience is also significantly enhanced with native applications.

    \item \textbf{Disadvantages:} The disadvantages include the potential for higher development costs and more complex maintenance for native applications.

  \end{itemize}

\textbf{non-native applications:} The non-native applications are designed so that the application can be used on any operating system.

  \begin{itemize}
    \item \textbf{Advantages:} The advantages include not having to create from scratch for implementation on another operating system, just adapting it. Additionally, the costs are lower than native applications.

    \item \textbf{Disadvantages:} The disadvantages include potential lower performance, and since it's not native, some functions may not be available for non-native languages.

  \end{itemize}
\end{document}
