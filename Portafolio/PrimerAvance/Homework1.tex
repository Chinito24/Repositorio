\documentclass{article}
\usepackage{graphicx} % Required for inserting images

\title{Homework 1 - Fundamentals of Mobile Development (Mobile Architecture)}
\author{Jorge Rafael Robledo Ramírez}
\date{Febrero 2024}

\begin{document}
\maketitle

The fundamentals of mobile development, especially in terms of architecture, involve how a mobile application is structured for its operation. Some of these points include:

\begin{itemize}
    \item \textbf{Client-Server Architecture:} Applications follow a client-server architecture, where the user communicates with a server to obtain data or send data that the user requires, and it executes on the device.

    \item \textbf{Mobile Operating System:} Devices utilize operating systems, such as Android or iOS, each with its own features and functions.

    \item \textbf{Native Development vs. Cross-Platform Development:} "In apps, there are native applications that are exclusively designed for a particular operating system and are developed using a specific programming language. In contrast, cross-platform applications are created using a different programming language and can be used on multiple operating systems."

    \item \textbf{User Interface (UI) and User Experience (UX):} The user interface primarily refers to the visual design that the user sees, while the user experience encompasses the interaction with the application.

    \item \textbf{APIs (Application Programming Interfaces):} APIs allow developers to utilize services from other companies to create more reliable and easily manageable software. APIs can include functionalities such as maps, comments, etc., as well as database-related data without directly interacting with the databases. They also provide access to certain information from companies.

    \item \textbf{Data Persistence:} Applications constantly use data that needs to be stored, and there are various methods for this, including embedded databases, local files, or cloud services.

    \item \textbf{Application Life Cycle:} Understanding the life cycle of a mobile application is crucial. This encompasses everything from initialization to suspension and closure of the application. Leveraging specific events in the life cycle allows for efficient resource management.

    \item \textbf{Testing and Debugging:} "The quality of the application is crucial; thorough testing should always be conducted to prevent faults and errors in the program, ensuring a positive user experience.
    
    \item \textbf{Updates and Maintenance:} Applications must undergo constant updates and maintenance, whether to add new content or to enhance aspects of both design and security.
\end{itemize}
\end{document}
